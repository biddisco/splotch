\documentclass[11pt]{article}
\usepackage{amssymb,amsmath}
\usepackage{graphicx} 

\title{Splotch on GPUs using the CUDA paradigm}
%\author{M. Rivi, C. Gheller, M.Krokos}
\author{.........}

\begin{document}
\maketitle

\section{Introduction}

The management and analysis of data generated by scientific experiments, 
observations and numerical simulations, currently represent an extraordinary challenge  
both for the research teams who concurred to its production and for 
the data centers, that have to provide technical solutions in terms of 
storage, computing power and software.

The main challenges are due both to the data size, aggregate but also of each 
single dataset, and to its complexity, useful information being hidden in a sea 
of bits. Standard data mining and analysis software often relies on complex
algorithms, that become prohibitively computationally expensive when dealing with 
huge datasets. Visual data exploration and discovery can represent a valuable
support, since they provide a prompt and intuitive insight to
very large-scale data sets to identify regions and/or features of interest within which
to apply time-consuming algorithms. 
Furthermore, this apporoach can be an extremely effective and ready way of discovering 
and understanding correlations,
similarities and data patterns, or to identify anomalus behaviors, that can be
interpreted as a wrong result, avoiding to waist time and effort on the specific dataset or
to save resources in an on-going experiment (e.g. stop a simulation that is producing 
unreliable outputs). Finally, visualization is also an effective means of presenting
scientific results both to experts and to general public.

In order to visualize huge datasets, suitable tools must be available, able to 
exploit High Performance Computing (hereafter HPC) devices, like multicore, multinode
supercomputers, providing suitable resources in terms of computing power, memory size, 
storage capacity and performance and network speed. Currently, not many such tools are avilable.

BIBLIOGRAPHY: visit and paraview, Australian stuff, Tipsy, visivo, stuff from MPI, 3D slicer, Splash

In this paper we focus on {\it Splotch} REF, our previously developed ray-casting
algorithm. Splotch was born for an effective high performance visualization of large-scale 
astrophysical data sets coming from particle-based computer simulations. The software is 
specialized in the high-quality, high-performance rendering of point like data as those 
produced in cosmology by N-Body numerical experiments, which 
represent prime examples of particle-based simulations. We can mention 
the Millennium ``trilogy'', the Horizon run or the DEUS simulation (REF) ETC., which
represent the most advanced and largest simulations in the field. 
This models reproduce the evolution of a meaningful fraction
of the universe by means of hundreds billion fluid elements, represented as particles,
interacting with each other through gravitational
forces. The typical size of a output (``snapshot'') of these simulations spans from several hundreds 
of Gigabytes up to tens of Terabytes, and typically stores the
ID, position and velocity together with additional properties, e.g.
local smoothing length, density and velocity dispersion, of each particle.

Splotch however, has been successfully adopted also in other application fields,
like the visualization of real galaxy systems, whose 3D shape is carefully modeled
according to observational data. Here, more than the data size, the driver is
the quality and the level of details of the final images,
that have to reproduce the full details of the
spectacular data coming from astronomical observations (REF in preparation). Furthermore,
it has been adopted also for the visualization of meshed based astrophysical simulations, 
although the same high quality cannot be achieved unless extremely high resolution
meshes are provided. 

In the development of Splotch, specific care has been taken of all the performance 
issues. The software is optimized in order to require the minimum possible
memory usage and in order to exploit vector architectures, multi-core processors 
and multi-nodes supercomputers (REF). 

In the last years, 
GPUs, and their computational counterpart, GPGPUs, have 
acquired more and more popularity both in the graphics and in the HPC 
communities, since they can provide extraordinary performances on suitable
classes of algorithms, with speed-up factors of about one order of magintude with respect to 
a standard multicore CPU and with a comparable power consumption.
Therefore, on HPC systems, graphic accelerators are becoming more and more common. 
Many supercomputers are equipped with hundreds of GPUs that can overlap 
their computing power to that of the CPUs, strongly reducing the time-to-solution
of many typical scientifc problems.

In order to exploit this additional computing resources, we have implemented 
a GPU version of Splotch. A full refactoring of the code has been necessary 
to get the good performance out of the GPU's implementation. 
All the details are presented in the rest of the paper. In Section 2, we give a short 
overview of the Splotch main algorithms. In Section 3, we recap the main features
of the GPU architecture and the CUDA progemming model
and we present the performance model that drove the 
re-design of the code. In Section 4 
we describe the implemented algorithm. Section 5 shows
the results of the performed tests and benchmarks. Finally, in Section 6, we draw 
the conclusions. 

\section{Splotch Overview}

The main peculiarities of the Splotch software are represented by the 
high quality of the generated images, obtained adopting a specific ray-casting 
approach, its high performance, due to a strong optimization on HPC architectures and the exploitation of multi-core, multi-node 
systems by means of an effective mix of the OpenMP and MPI parallel programming paradigms, and
the support of large data volumes, through an optimal usage of the memory, with
no data replica or unnecassary allocations, a full 64 bits support and the exploitation
of both shared and distributed memory systems. Furthermore, Splotch's design is kept 
intentionally simple, making its extension to new components and functionalities easy. 
Finally, the software package is completely self-contained, with no dependencies from external 
libraries (apart from those needed for parallelism and HDF5 - REF - for
specific input data file formats). The code is fully C++ and its compilation
usually straightforward, as soon as a suitable makefile script is provided.
No configure procedure is currently supported.

The Splotch algorithm operational scenario consists of a number of stages. 
\begin{itemize}
\item
Data load: data is read from one or more data files. A number of different file
formats are supported, from raw binaries, to HDF5 to more specific Gadget (REF) or Tipsy (REF)
files. The $x$, $y$ and $z$ coordinates are the only compulsory quantities to provide.
These can be cartesian geometric coordinates, but also any other tern 
of variables, adopted to explore a generic three dimensional parameter space.
\item
Geometry set-up: particle coordinates and other geometric quantities 
(like smoothing lengths, see below) are roto-translated and projected according to the 
camera configuration (camera position, look-at direction and orientation). 
Active particles (that actually contributes to the image according to the point of view)
are identified. All the following operations are performed only on active particles.
\item
Data processing: data is normalized, necessary 
transformation (e.g. calculations of logarithms of processed quantities) are performed
and RGB colors (according to read data or to a given
external color table) are associated to each data point. 
\item
Ray-casting: particle contribution to image pixels are calculated.
\item
Image save: final images are saved using the TGA high quality format.
\end{itemize}
The workflow is shown in Figure , where the parts of the algorithm parallelized
either with MPI or with OpenMP are emphasized. For a detailed description of
the hybrid parallel implementation of Splotch and its performances, we refer to REF.

%\begin{figure}
%\centering
%\includegraphics[scale=0.5]{Images/workflow.jpg}
%\caption{...}
%\label{fig:workflow}
%\end{figure}

The most peculiar part of the Splotch software is represented by its ray-casting algorithm.
This is also its most computationally demanding component.
In order to effectively and efficiently handle point-like distributions, the following 
steps are carried out: 

\begin{itemize}
\item
the particle $p$ with coordinates $\vec r$ transports a quantity $\rho_p(\vec r)$ 
(e.g. the mass density or the temperature at position $\vec r$)
that modulates the contribution of the specific particle to the image
according to the following gaussian distribution: 

\begin{equation}\label{smooth}
\rho_p(\vec r)=\rho_{0,p}\exp(-r^2/\sigma_p^2),
\end{equation}

where $\sigma_p$ is the smoothing length, which represents
a sort of size of each particle.  
In principle, each data point contributes to all the pixels of the final image.
In practice, it is much more handy to have a compact support of the
distribution, that is, the distribution is set to zero at a given
distance $f\cdot\sigma_p$, where $f$ is a proper multiplicative factor.
Therefore rays passing
the particle at a distance larger than $f\cdot\sigma_p$ are
unaffected by $\rho_p$.

\item
Three ``frequencies'' to describe the red, green and blue
components of the radiation, respectively, are used. These are treated independently.

\item
The radiation intensity $\bf{I}$ (treated
as a vector with r, g and b components) along a ray through the simulation
volume is modeled by the well known radiative transfer equation
\begin{equation}\label{rad}
\frac{d\bf{I}(x)}{dx}=(\bf{E}_p-\bf{A}_p\bf{I}(x))\rho_p(x),
\end{equation}
which can be found in standard textbooks REF.
Here, $\bf{E}_p$ and $\bf{A}_p$ describe the strength of radiation emission and absorption
for a given particle for the three rgb-colour components. 
%In general it is recommended to
%set $\bf{E}_p=\bf{A}_p$, which typically produces visually appealing images. This is presently a
%necessary setting for Splotch, in order to reduce the complexity of some aspects of its parallel
%implementation. This constraint will be eliminated in the next releases of the code.
If a scalar quantity is chosen (e.g.\ the particle temperature,
density, velocity dispersion, etc.), the mapping to the three components of $\bf{E}$ and $\bf{A}$ (for red, green and blue)
is typically achieved via a transfer function, realized by a colour look-up table or palette, which can
be provided to the ray-tracer as an external file to allow a maximum of flexibility. If a
vector quantity is chosen (e.g.\ velocity, magnetic field, etc.), the three components of the vectors
can be mapped to the three components of $\bf{E}$ and $\bf{A}$ (for red, green and blue). 
In addition
to the color, the optical depth of each particle can be also modulated proportionally to another
scalar property (e.g.\ density, etc.).
\end{itemize}

Further details on the Splotch rendering algorithm can be found in REF.

\section{The GPU Code}

GPUs can represent an effective tool for Splotch to dramatically 
reduce the time-to-solution and to step toward real-time interaction.
However, the real effectiveness of GPUs on Splotch rastering algorithm has to
be carefully analysed and estimated. Splotch, in fact, poses serious challenges 
to the GPU's programming model, that privileges highly data parallel algorithms 
where the hundreds of cores of the GPU can work independently from each other.
Any kind of dependency is strongly penalized, reflecting on the final performance.
Splotch, however, violates this basic requirement in the following aspect:
each particle can affect a different number of pixels, depending both on its intrinsic
smoothing radius and on the camera position (the same particle 
can affect none or all the pixels of an image, depending on the point of view). 
Hence, an optimal load balancing can be hard to achieve 
and, even worse, it can lead to frequent concurrent accesses to memory, since
different threads acting on different particles may affect the same pixel,
leading to wrong results.

The potential dependency of any pixel of the image from any data point,
makes the distribution of work between GPU's cores challenging to implement
and even harder to tune in terms both of performances and of memory requirements.
In section XXX, we propose a simple performance model that has
driven the implementation of the GPU enabled Splotch kernels.

\subsection{The CUDA programming model} 

For the implementation of the GPU code we have adopted the CUDA programming model (REF). This is currently the standard ``de facto'' for GPU programming.
Its main drawback is represented by the limited portability, being an NVIDIA product. However,
this ensures that the CUDA programming model closely maps and supports the underlying
hardware (NVIDIA cards), leading to an optimal tuning of the performance.
The CUDA standard offers access to highly parallelized modern GPU architectures via a simplified C/C++ or Fortran language interface. It is designed to support joint CPU/GPU execution of an application, where serial sections are performed by the CPU (host), while those which exhibit rich amount of data parallelism, are performed by the GPU (device) as CUDA kernels. The CPU and the GPU have separate memory spaces so data must be transferred from each other via PCIe bus. CUDA kernels launch is asynchronous, so that the host can
 perform the following instructions of the code while the device is computing. If one of them requires the completion 
of the kernel execution, then it is necessary to call the \textit{cudaThreadSyncrhonize}() function before executing it.
A kernel is written for a single thread and instantiated as a grid of many lightweight parallel threads, organized into blocks of the same size. A thread is an indipendent element of work and maps to a hardware core. A block is a 1D, 2D or 3D set of concurrently executing threads that can cooperate among themselves through barrier synchronization and "fast" shared memory. 
This is possible because threads belonging to the same block are executed on the same multiprocessor (SM). On the other hand
 synchronization is not possible between blocks of a grid. In fact, the limited amount of memory limits the number of 
blocks that can simultaneously reside in the same SM. Moreover when one block stalls the runtime system switch to 
a different one, hence there is no guaranteed order of execution.

Once a block is assigned to a SM, it is partitioned into 32-thread units called warps. They are scheduling units in SM:
all threads in a same warp execute the same instruction (Single-Instruction, Multiple-Thread). Hence, programmers should minimize the number of execution branches inside the same warp. It is convenient to assign a large number of warps to each SM (i.e. high occupancy), because the long waiting time of some warp instructions is hidden by executing instructions from other warps  and therefore the selection of ready warps for execution does not introduce any idle time into the execution timeline (zero-overhead thread scheduling). Further details on CUDA can be found in (REF).

\subsection{The Performance Model}

% probabilmente questa e' una ripetizione
Splotch's main computational kernels are the {\it Normalization}, re-calculation
of the different quantities in proper units, {\it Geometry}, roto-translation 
of the reference frame, {\it Coloring}, assignment of the RGB colors associated to each 
particle,  {\it Rendering},
rasterization and ray casting, 
and {\it Image Creation}, composition and save of 
the final image. The initial data load phase, though often demanding, will not be considered
since it is not subject of the GPU implementation.

The main parameters of the model are $N_{part}$, the number of particles
to be processed, and $N_{pix}$, the number of pixels of the final image
(in general we can consider square images with $N_{pix}=N_{pix,x}^2$). For 
the Splotch's rendering kernel a third parameter must be introduced, 
that is the average smoothing length of the particles,
$R_s$. Its value depends both on the intrinsic "size" of the particle
and on the position of the point of view (particle size increases getting closer to 
the point of view).

%The kernels scale with these parameters as:
%\begin{align}\label{scaling}
%& N_{Norm} \propto N_{part},\\
%& N_{Geom} \propto N_{part},\\
%& N_{Color} \propto N_{part},\\
%%& N_{Sort} \propto N_{part}{\rm log}(N_{part}),\\
%& N_{Render} \propto N_{part} R_s^2,\\
%& N_{Image} \propto N_{pix,x}^2,
%\end{align}
%where $N_{kernel}$ is the number of operations for a given kernel.

%The Normalization, Geometry and Coloring kernels are  
%perfectly data parallel, each particle being processed independently from the others.
%Thus they are expected to fit the GPU programming model. 
%The Rendering kernel is, in general, the most computationally demanding. 
%The computational time depends in a predictable way with the number of particles and the number 
%of pixels. The most challenging dependecy, however, is from the smoothing length. 
%This dependency leads to strong changes in the computational effort and makes
%the work-load hard to balance, when this is distributed among many-processors
%systems.

In order to model the Splotch algorthm's performances, we split the different timings
required to perform the main computational steps. 
The Splotch's performances can be quantified as time spent on the CPU, time spent on the GPU
and time spent to move data among different memories:
\begin{equation}\label{Ts}
T_{TOT} = T_{cpu} + T_{gpu} + T_{pci} + T_{Mgpu},
\end{equation}
where $T_{TOT}$ is the total time, 
$T_{cpu}$ is the time spent on the CPU, $T_{gpu}$ is the time
spent on the GPU for computation, $T_{pci}$ is the time needed to move data from
the CPU to GPU and back through the PCI Express bus and $T_{Mgpu}$ is the time 
spent in moving data among the global and the shared memory on the GPU.

Time spent in transferring data
between CPU and GPU, $T_{pci}$, is typically the main 
bottleneck in GPU usage. Therefore the amount of data transferred from host
to device and back has to be minimized. 
All particle data has to be offloaded to the GPU. If $S_{part}$ is the
size of a single particle (in the current implementation $S_{part}=35$ bytes),
$N_{part} S_{part}$ is the total amount of particle data moved to the device.
The final processed image (the result) is the only data that has to be copied back to
the CPU. Thus, in principle, only $3 N_{pix,x}^2$ bytes (R, G, B values) have to be transferred at
the end of the rendering phase. However, further data transfers may be necessary
during the computation. Such data movements should be hidden
by overlapping them with calculation, exploiting CUDA asyncronous operations.

Particles' data offload can be accomplished either in a single or in a few operations, 
depending on the data size (if it fits the available GPU memory) and/or
specific features of the algorithm (in some cases it may be convenient to
split data into chunks, copying on the GPU one chunck after the other).
In general we can estimate the data transfer time as:
\begin{equation}\label{pci}
T_{pci} =  N_{chunks} \tau_{pci} + {N_{part} S_{part} + 3 N_{pix,x}^2 \over 
\mu_{pci}},
\end{equation}
where $\tau_{pci}$ is the transfer time latency (in seconds) and $\mu_{pci}$ is the
bus bandwidth (in bytes per second), $N_{chunks}$ is the number 
of copy stages. 

The time spent in processing the particles on the GPU can be estimated as:
\begin{equation}
T_{gpu} = N_{op}/\nu_{GPU},
\end{equation}
where $\nu_{GPU}$ is the GPU flops/sec rate and
\begin{equation}\label{ops}
N_{op} = N_{part}(\alpha + \gamma R_s^2) + f_{GPU},
\end{equation}
with $\alpha$ and $\gamma$ estimating the number of operation of 
the combined Normalization, Geometry and Coloring kernels,  
and of the Rendering kernel respectively. The function 
$f_{GPU}$ takes into account any GPU specific part of the algorithm. 

Global memory accesses can be estimated 
as the number of loads from the global to the shared memories of the available 
streaming multiprocessors plus the number of stores from the shared to the global memory: 
\begin{equation}
N_{Mgpu} = (N_{load,p} + N_{store,p}) N_{part} + N_{store,pix} N_{pix,x}^2,
\end{equation}
where $N_{load,p}$ and $N_{store,p}$ are the number of load and store of the 
particles, while $N_{store,pix}$ is the number of stores of the image (no loads 
are expected, since the image is created on the GPU). 
The time for memory access is:
\begin{equation}\label{tmgpu}
T_{Mgpu} = {(N_{load,p} + N_{store,p}) N_{part} S_{part} 
+ 3 N_{store,pix} N_{pix}^2\over \mu_{gpu}}
+ g_{GPU},
\end{equation}
where $\mu_{gpu}$ is the global memory bandwidth and $g_{GPU}$ the time 
spent on memory accesses by the GPU specific
functions, corresponding to the $f_{GPU}$ term in equation \eqref{ops}.
The terms $f_{GPU}$ and $g_{GPU}$ will be discussed in details in section XXX.

The final contribution to equation \eqref{Ts} is $T_{cpu}$. In principle this should be negligible, 
the large part of the work being performed by the GPU. Actually, the CPU
could contribute in processing a proper fraction of the particles. Using
asyncronous operations this work can be overlapped 
to the GPU work, as described in section XXX.

\subsection{Performance Model analysis}

Three classes of parameters characterize the performance model.
the number of particles $N_{part}$,
the image size $N_{pix,x}^2$ and the characteristic particle size $R_s$ belong to
the first class. They 
are ``model" parameters, related to the specific dataset or to the image quality.
A second class of parameters is related to the algorithm and the way 
it has to be designed in order to exploit the architectural characteristics
of the GPU. The identified parameters are $N_{chunks}$, $N_{load,p}$, 
$N_{store,p}$,  and $N_{store,pix}$. Furthermore, $f_{GPU}$ and 
$g_{GPU}$ have to be properly described and their impact on the performances
estimated (see sections XXX and YYY). The final class of parameters 
is specific to the GPU architecture. In our analysis we have set them to fiducial
values related to the NVidia M2090 Fermi (REF) GPU, that was used for our
tests and benchmarks. More specifically: 
$\mu_{gpu} = 177$ GB/sec, $\tau_{pci} \sim 10^3$ nsec, $\nu_{gpu} = 665$ GFlops/sec and
$\mu_{pci} \sim 6$ GB/sec.
A further set of parameters, like the number of blocks, the number of threads per block etc., 
is related to the CUDA progeamming model. 

Equations $\eqref{pci}$ and $\eqref{tmgpu}$ show that performances
depends linearly from both the number of particles and the number of pixels. 
In the case of large datasets we have that
$N_{part} >> N_{pix,x}^2$. For instance, the test case adopted 
in section XXX consists in $N_{part} \sim 10^8$
and $N_{pix,x}^2 \sim 10^6$. 
In that case, the contribution of $N_{pix,x}^2$ 
to $T_{pci}$ is negligible. 
The term $T_{Mgpu}$ appears to have a comparable behaviour. However, 
this holds only if all the particles falls in the field of view (so in
the rendered scene),  unless the 
parameter $N_{store,pix}$ is of the order of 100 or larger. Hence, the algorithm has to
be designed such that the number of copies of the image's pixels to the 
global memory is $O(1)$. At that point, all the dependencies on the images size can
be neglected. If instead the point of view is placed such that part of the particles 
falls outside the field of view (e.g. inside the particle distribution), these 
particles do not contribute to the calculation and the two terms of equation 
\eqref{tmgpu} can become comparable.

Equation \eqref{ops} shows that $N_{op}$ has a critical dependency from 
$N_{part} R_s^2$. The dependency from the product of these two parameters, makes 
this contribution the most important from a computational point of view and the most
tricky to model and to tune. This is due especially to the unpredictability of
both terms, that depend on the position of the point of view. In particular, 
the $R_s$ parameter can lead to a strong increase of the computing time, since 
it can be large, involving many pixels and huge unbalances in the processing 
time of each single particle. This
represents a big issue in the case of parallel processing, as in a multi-core or
many-core system, where a good workload balance is crucial to achieve good performances.
A further difficulty associated to a changing smoothing length emerges when 
multithreading operations on particles are exploited, as expected on a GPU. 
In this case, the same pixel could be updated concurrently by different threads,
with a consqeunt loss of one or more contributions.
The quality of the resulting image is therefore degraded and its correctness may be
compromised.
Specific solution must be designed in order to circumvent this problem 
without paying big penalties in terms of performance (as would happen, for instance,
adopting atomic updates, supported by CUDA, but too slow to be used for continuous 
updates of the pixels values).

Data offload to the GPU is one of the crucial steps in the algorithm. 
Equation \eqref{pci} shows that we have a minimum amount of data 
to be moved to the GPU: the particles information (coordinates, fields, smoothing length). This leads to an 
unavoidable overhead, that, however, can be masked by overlap with computation using
CUDA asynchronous data transfer. In order to implement this solution data have to be
splitted in chunks and passed to the GPU one after the other, in a loop. Whilst chuncks 
are transferred, computation on previous chuncks, already stored on the GPU
memory, can be carried out. Splitting data in sub-chunks can also be necessary
in order to fit data in the GPU global memory, since the whole dataset could exceed
that available on the device. According to equation \eqref{pci}, this does not 
cause any meaningful overhead if the copy latency is kept small, that is:
\begin{equation}
N_{chunks} < {N_{part} S_{part}\over \tau_{pci}\mu_{pci}}.
\end{equation}
For instance, if $N_{part}$ is $O(10^8)$, $N_{chunks}$ can be up to $10^5$ before
contributing to the overhead.

Equation \eqref{tmgpu} show that the performance related to global memory access
depends on the two parameters $N_{load,p}$ and $N_{store,p}$, that quantify the 
total amount of data transferred from the global to the shared memories and back,
whose performance depends on the memory bandwidth.
The general rule is to keep these parameters
as small as possible. 
The optimal solution would be to have one thread 
processing one particle. In this case, we would have $N_{load,p} = 1$ and $N_{store,p} = 0$.
This solution, though ideal, is not practical, the main reason being the frequent race conditions 
rising from threads trying to concurrently update the same pixels.
Only part of the algorithm is data parallel and can adopt the ``one thread per particle" solution.
This corresponds to the Coloring and Geometry kernels, in which each particle 
is processed independently from the others. Merging this two components in a single
kernel, we have $N_{load,p} = 1$ and, since processed particles
have to be copied back to the global memory to be rendered, $N_{store,p} = 1$. 

\section{The CUDA implementation}

As soon as data is loaded in the global memory of the GPU, each particle is processed in order to have 
its coordinates transformed into screen coordinates (rasterization - Geometry kernel) 
and the colours assigned (Coloring kernel). In both kernels, each particle can be computed 
independently from the others, therefore, as already pointed out in section XXX, 
a ``one-thread-per-particle" approach can be adopted, with full exploitation of
the GPU architecture. In order to increase the efficiency of the algorithm, the two kernels 
have been fused in a single ``Set-up" kernel, 
so that the number of operations per memory access is maximum. 
At that point, particles are ready to be rendered (Rendering kernel).
This requires a carefully implementation, mainly due to the two main issues emerged 
in the performance modelling analysis:
\begin{itemize}
\item
race conditions writing different contributions to the same pixel,
\item
workload unbalace between different processing units,
\end{itemize}
Such issues make the simple ``one-thread-per-particle" approach unsuitable. Other simple 
approaches, tohugh possible, cannot be adopted in order prevent the performances to drop dramatically.

\subsection{Rendering algorithm design}

The rendering algorithm gas been designed for the GPU according to the following procedure.

Particles are first classified in 3 groups according to their size:
\begin{itemize}
\item 
C1: $r > r_0$
\item
C2: $1 < r < r_0$
\item
C3: $r < 1$
\end{itemize}
where $r$ is the particle smoothing length in pixels, which depends both on the 
intrinsic properties of the particle (the ``physics'' of the problem) 
and on the camera position, and $r_0$ is a fixed threshold. Each class of 
particles is rendered in the most convenient and efficient way by making CPU and GPU 
work concurrently.

Particles of different classes pose different challenges to the rendering 
algorithm and can be treated with different approaches. Particles of class $C3$ 
can be assumed to influence a single pixel, therefore they can be efficiently 
processed with a ``one-thread-per-particle" approach. Setting $r_0 \sim 0.01 N_{pix,x}$, 
particles of class C2 can influence only small fractions of the image. If 
we divide the image in \``tiles" of suitable size, each C2 particle can affect 
only one tile (see below for details). Finally, particles of class C1 can affect 
a large fraction of the image, spanning multiple tiles.

The particle classification is performed in the Set-up kernel, with negligible impact
on the overall performance. Each particle is labeled
with an a {\it tile index} $n_c$ as follows:
\begin{itemize}
\item 
$n_c = -2$, inactive particle (outside the field of view);
\item
$n_c = -1$, if it belongs to C1 class; 
\item
$0 < n_c < N_t$, if it belongs to C2 class and its centre falls in the $n$-th tile of the image;  
\item
$n_c = N_t$, if it belongs to C3 class, i.e. is point-like.
\end{itemize}
The parameter $N_t$ is the number of tiles the image is divided in. It is defined as
\begin{equation}
N_t = {N_{pix,x} N_{pix,y}\over (2r_0)^2}.   
\end{equation}

After this operation, a new kernel has to be implemented, contributing to the $f_{GPU}$
and $g_{GPU}$ terms of equations \eqref{ops} and \eqref{tmgpu} respectively. 
Within the new kernel particles are sorted by the $n_c$ key and the number 
of particles with the same index is calculated. At the end of this operation, 
all particles with the same tile index are contiguous in memory, and can be easily 
isolated for different kinds of processing. 

The sorting operation plays a crucial role. It allows to manage particles on the 
device and it is necessary for the efficient execution of further operations like 
reduction and prefix sum by key. In fact, memory accesses to the elements involved 
in a CUDA operation are optimized when all of them are consecutives. However, sorting 
is intrinsically an expensive operation. Hence, an efficient CUDA implementation of
the sorting function has to be adopted in order to reduce its overhead. 
The Thrust library (REF) provides such function. Thrust is a C++ template library 
for CUDA which mimics the Standard Template Library (STL) and provides 
optimized functions to manage very large arrays of data. In particular, Thrust 
implements a highly-optimized Radix Sort algorithm (REF) for sorting primitive types
(e.g., char, int, float, and double) with the standard less comparison operator and 
apply dynamic optimizations to further improve its performance.

The \textit{tiling scheme} for rendering C2 particles on the device consist in 
assigning particles related to the same image tile to a block of CUDA threads 
and exploit the shared memory and thread synchronization within the block to store 
and compose the image tile. The tile side size is set so that any particle 
belonging to the tile is entirely contained in it. This is achieved by 
defining a $body$ of the tail of $2r_0$ pixels and $edges$ of $r_0$ pixels on each side 
of the tile (see figure XXX). 
We will refer to this extended tile as a \textit{Btile}).
Particles are accessed in chunks of YYYYYYYYYYYYYYY  by thread 0 of each block and stored 
in the shared memory so that all threads of the same block can access their data. 
Then each pixel of the current particle is rendered by a different thread of 
the block (the pixel number processed by each thread may change as the 
particle varies, but it remains in the Btile). 
This solution avoids race conditions when composing the image tile, since each 
thread of the same block accesses different pixels and the workload of each thread 
(except thread 0) is the same even if particles may have different size.

When all particles of the block are rendered, the contribution of the Btile 
is added to the image stored in the global memory. First, each thread adds
$4r_0^2/block\_size$ pixels of the body of the Btile. For these pixels no race 
conditions occurs: each block accesses different tiles and each thread accesses 
different pixels of the tile. The edges must be added in a separate step. 
Concurrences among the boundary of a tile and the adjacent ones occur (see
figure XXX) between:
\begin{itemize}
\item
a boundary and the body of two different Btiles (2 concurrences), 
\item
a corner and the body of two Btiles, the horizontal boundary 
of a third Btile, the vertical boundary of a fourth Btile (4 concurrences). 
\end{itemize}
%CLA: DA RIVEDERE CON LA FIGURA...
As a solution we implemented the following 3 steps (see Figure ...):  
first add a and b sides (no race conditions between boundary sides of tiles nearby),
then add b and c sides (no race conditions between boundary sides of tiles nearby),
finally add corners (no race conditions between corners). 
Note that the number of pixels of the 4 boundary corners of a tile are equal to 
the size of a particle, i.e. the size of a block. Since CUDA blocks are order 
independent, we need to add these contributions to 3 different copies of the image 
to avoid race conditions. Then, at the end of the execution of the render kernel 
it is possibile to add these image copies to the final one.

Rendering of C3 particles simpler, since they are affect a single pixel. 
Thus, only the position of the particle in the global image has to be calculated.
This can be efficiently carried out by assigning a CUDA thread to each C3 particle. 
However, since different threads can affect the same pixel, their contribution (fragment) 
cannot be directly added to the image. Hence, this is done in a second step by allocating 
in the device memory one buffer (fragment buffer) to storing fragments and
corresponding index ids. Finally, the image is produced by reducing by key (pixel id) the 
fragments. This consists in a sort of the fragments by key, followed by the reduce of 
contiguous elements falling in the same pixels, using the corresponding 
functions provided by the Thrust library.

For both C2 and C3 particles the main drawback of the proposed solutions 
is represented by the overhead due to the sorting and reduction operations.
%CLA: QUA CI VUOLE QUALCHE COMMENTO IN PIU'
Performances of the C2 particles rendering algorithm are also influenced 
by the size of the shared memory, which limits the number of resident blocks
per multiprocessor during the execution of the kernel, thus reducing the theoretical
occupancy. A further issue is related to the
unbalanced work load of each block. In fact the number of particles falling
in each tile can vary in a significant way from one tile to another. 
For C3 particles, a limiting factor is related to the size of the fragment buffer,
that cannot exceed the free memory space. This, however, can be 
easily overcome by splitting the proccessing in a number of iterations suitable 
to the amount of available memory.  
All these aspect will be quantitatively analyzed in section XXX.

Particles of class C1 are the most challenging to process. Their large smoothing 
radius prevents the usage both of a tiles based solution and of a fragment buffer.
In the first case, tiles would be too large to be stored in the shared memory
and, in all cases, their large overlap could lead to strong overheads in the reduction of the 
final image. A fragment buffer instead could require too much memory, since 
for each C1 particle a great number of fragments would be generated, possibly
larger than the available memory. The adopted solution circumvented these 
difficulties by copying C1 particles back to the CPU and perform the rendering 
with the original, serial algorithm. This is possible thanks to CUDA asynchronous
operations, which allows to copy data from the device to the host (and vice-versa)
whilst the calculation on the GPU proceeds. Once C1 data are back on the CPU, 
their processing can be performed concurrently to that of the GPU.  
With this approach, we manage to exploit both the host and device
at the same time, hiding the difficulties related to the rendering of ``big"
particles. Obviously, this solution is effective as long as the number of C1 particles 
is much smaller than that of particles belonging to the other two classes. 
In the worst case, all the particles would be C1 and the time to solution would be slightly 
longer than that of the pure sequential code. In section XXX we discuss
detailed figures related to this aspect.

The final step is the composition of the two partial images (one processed by the
GPU the other by the CPU) into the final result. 
Such operation is performed by the CPU, 
once the GPU partial image has been transferred, with a single copy operation, to
its memory. The data transfer has no impact on the performance, involving just a few 
megabytes of data. The same holdes for the reduction of the two partial images.   

\section{Tests and Results}

We have analyzed the performances of Splotch's GPU implementation on a number of cases, designed in 
order to stress the different features of the algorithm, presented in the previous 
sections, comparing the results 
obtained using the GPU to those obtained with the original code on the CPU. 

The dataset adopted in all the tests is the result of a medium sized cosmological 
N-body simulation performed using the Gadget code (REF). It consists in about 
400 million particles, with about 200 million of dark matter, the same amount 
of baryonic matter (herafter ``gas'') and about 10 million star particles. 
Particles of all species are characterized by their spatial coordinates, velocities
and smoothing length. 
Both the gas and the stars have a number of other quantities associated, that can be used 
as color and intensity. Mass density and temperature for the former, spectral type and YYYYYY
for the latter. Dark matter will not be used in our tests.

The large size of the dataset, $\approx 7.5$ GB can be handled only by a large memory computing node. 
This is not an issue for the GPU, since data are splitted 
in smaller chunks and loaded iteratively on the accelerator. 

Our tests have all been performed on a IBM dx360 M3 node, based on the dual-socket 
six-cores Intel Xeon 5650 processor architecture running at 2.6 GHz, 
offering 24 GB of main system memory per node. The node is equipped with 2 NVIDIA 
Tesla FERMI GPUs per node of type M2090 (REF), with 6 GB of global memory, 177 GB/sec 
main memory bandwidth, 665 GFlops/sec of peak performance. GCC 4.4.5 and CUDA 4.2 
where used for compiling the code.
%CLA: AGGIUNGI I DATI CHE RITIENI UTILI

\subsection{GPU performance tuning}

%CLA: QUI METTEREI TUTTO CIO' CHE E' LEGATO AI TEST CHE HAI FATTO PER CUDA

\subsection{Performance analysis}

The main tests' parameters are the data size ($N_{part}$),
the image size ($N_{pix,x}^2$) and the distribution of the number of pixels 
per particle ($N(r_s)$), which is influenced by the position of the point of view.
This last parameter is related to $R_s$ such that $R_s = <r_s>$ this last term 
being the expectation value of the smoothing radii, $r_s$, of the particles.

............


\section{Conclusions and Next Steps}

\end{document}
