\documentclass{report}
\parindent0cm
\parskip0.5cm
\begin{document}

{\bf \large Equations governing intensity calculations in {\tt splotch}}

Only a single kind of intensity is considered. For full RGB, this has to be
applied to all color components.

Definitions:
\begin{equation}
Q_p := \frac{E_p}{A_p}
\end{equation}
\begin{equation}
R_p := \exp\left(\int_{-\infty}^{\infty}\rho_p(x)dx\right)
\end{equation}

Intensity change caused by a single particle (general case, corresponds to
equation 3 in the original Splotch paper):
\begin{equation}
I_2-I_1=\left(e^{-A_p R_p}-1\right)(I_1-Q_p)
\end{equation}

{\bf Special case:} $Q_p=$const$=:Q$ independent of particle
\begin{equation}
I_2-I_1=\left(e^{-A_p R_p}-1\right)(I_1-Q)
\end{equation}
\begin{equation}
\frac{I_2-I_1}{I_1-Q} + 1=\frac{I_2-I_1 + I_1-Q}{I_1-Q}=\frac{I_2-Q}{I_1-Q}=e^{-A_p R_p}
\end{equation}
\begin{equation}
\ln|I_2-Q|-\ln|I_1-Q|=-A_p R_p
\end{equation}

Variable substitution: $Y:=-\ln|I-Q|$
\begin{equation}
Y_2-Y_1=A_p R_p
\end{equation}
Instead of $I$, the rendering algorithm computes the resulting $Y$
for every pixel.
At the end, the intensity is obtained by
\begin{equation}
I=Q-e^Y
\end{equation}
If we assume that $Q=1$ (this is currently assumed for {\tt a\_eq\_e} in Splotch),
then the starting value for $Y$ is 0 in every pixel.
For the somewhat more general case of $Q=$const, but different from 1, the starting
value for $Y$ would have to be $-\ln|Q|$.

As a consequence, the quick rendering algorithm can be extended to scenes where
$E/A$ is constant, but not necessarily 1.
\end{document}
