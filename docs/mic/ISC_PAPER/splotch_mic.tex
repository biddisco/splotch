
%%%%%%%%%%%%%%%%%%%%%%% file typeinst.tex %%%%%%%%%%%%%%%%%%%%%%%%%
%
% This is the LaTeX source for the instructions to authors using
% the LaTeX document class 'llncs.cls' for contributions to
% the Lecture Notes in Computer Sciences series.
% http://www.springer.com/lncs       Springer Heidelberg 2006/05/04
%
% It may be used as a template for your own input - copy it
% to a new file with a new name and use it as the basis
% for your article.
%
% NB: the document class 'llncs' has its own and detailed documentation, see
% ftp://ftp.springer.de/data/pubftp/pub/tex/latex/llncs/latex2e/llncsdoc.pdf
%
%%%%%%%%%%%%%%%%%%%%%%%%%%%%%%%%%%%%%%%%%%%%%%%%%%%%%%%%%%%%%%%%%%%


\documentclass[runningheads,a4paper]{llncs}

\usepackage{amssymb}
\setcounter{tocdepth}{3}
\usepackage{graphicx}

\usepackage{url}
\urldef{\mailsa}\path|{timothy.dykes, mel.krokos}@port.ac.uk|    
\newcommand{\keywords}[1]{\par\addvspace\baselineskip
\noindent\keywordname\enspace\ignorespaces#1}

\begin{document}

\mainmatter  % start of an individual contribution

% first the title is needed
\title{Big Astrophysical Data Visualisation on the MIC architecture}

% a short form should be given in case it is too long for the running head
%\titlerunning{Lecture Notes in Computer Science: Authors' Instructions}

% the name(s) of the author(s) follow(s) next
%
% NB: Chinese authors should write their first names(s) in front of
% their surnames. This ensures that the names appear correctly in
% the running heads and the author index.
%
\author{Timothy Dykes\inst{1}
\and Claudio Gheller\inst{2}
\and Marzia Rivi\inst{3}
\and Mel Krokos\inst{1}
}
%
%\authorrunning{Lecture Notes in Computer Science: Authors' Instructions}
% (feature abused for this document to repeat the title also on left hand pages)

% the affiliations are given next; don't give your e-mail address
% unless you accept that it will be published
\institute{   
   University of Portsmouth,
   Portsmouth, U.K.\\
\mailsa\\
\and
  CSCS-ETHZ,
  Lugano, Switzerland\\
  \email{cgheller@cscs.ch}
\and
   University of Oxford,
   Oxford, U.K.\\
   \email{rivi@physics.ox.ac.uk}\\
%\mailsb\\
%\mailsc\\
%\url{http://www.springer.com/lncs}}
}
%
% NB: a more complex sample for affiliations and the mapping to the
% corresponding authors can be found in the file "llncs.dem"
% (search for the string "\mainmatter" where a contribution starts).
% "llncs.dem" accompanies the document class "llncs.cls".
%

\toctitle{Lecture Notes in Computer Science}
\tocauthor{Authors' Instructions}
\maketitle


\begin{abstract}
The abstract should summarize the contents of the paper and should
contain at least 70 and at most 150 words. It should be written using the
\emph{abstract} environment.
\keywords{We would like to encourage you to list your keywords within
the abstract section}
\end{abstract}


\section{Introduction}
\label{sect:introduction}

The scientific data volume produced by experiments, observations and numerical simulations 
is increasing exponentially with time. This is true for any scientific domain, but in 
particular for astrophysics. Next generations of telescopes and antennas are expected to produce
enormous amount of data. The Square Kilometer Array project (REF), for instance, will 
generate an exabyte of data every day, twice the information currently exchanged on 
the internet on a daily basis and 100 times more information then the CERN LHC (REF) experiment
produces. At the same time, computer simulations represent an invaluable instrument for 
astrophysicists to validate theories and compare to observations through numerical experiments.
The ultimate cosmological simulations (REF), performed using sophisticated N-body codes (REF), 
could describe the details of the evolution of the universe up to the present time, following the behaviour of 
gravitating matter represented by a hundred billion particles. These runs produce output files 
whose size is of the order of tens of terabytes each. The fast technological progresss 
of supercomputing systems will soon lead to simulations producing outputs with size of the order of 
the petabyte or more. 

Size does not represent the only challenge posed by scientific data. Is also essential to 
effectively extract all the information hidden in the sea of bytes represented by each 
single data file. Software for data mining and analysis is often highly computationally
demanding and ultimately unusable on large datasets.  

Visual exploration and discovery represents an outstanding aid to big data processing, 
e.g. by providing scientists with
prompt and intuitive insights enabling them to identify interesting 
characteristics and thus define regions of interest within which to apply time-consuming 
methods. Additionally, they can be a very effective way in discovering and
understanding correlations, associations and data patterns, or in identifying unexpected behaviours
or even errors. Visualization is also an effective means for communicating scientific results not
only to researchers but also to members of the general public.
However, also visualization tools require high performance computing (hereafter HPC) resources, to 
fulfil the requirements posed both from data size and from the need of having
fast (if not real-time) rendering.

The Splotch software (REF), our ray-casting algorithm for
effectively visualizing large-scale, particle-based datasets, addresses these issues, 
providing high quality graphic outputs, processing data of, ideally, any size, already
efficiently exploiting a broad variety of HPC systems: multi-core processors and multi-node supercomputing systems 
(REF: Procedia Computer Science, 1(1) pp.1775-1784, 2010), and GPUs (REF1: Astronomical Society of the Pacific Conference 
Series, 475 (ADASS XXII) pp.103-106, 2013; REF2: in preparation). 
This paper will describe the work accomplished to enable Splotch to 
run on the new Intel PHI (REF) accelerator, taking advantage of the Many Integrated Core (hereafter MIC) architecture,
which is expected to provide, on suitable classes of algorithms, outstanding performance 
with power consumption being comparable to standard CPUs. 
We will present the MIC implementation and optimisations, performance tuning,
benchmarks carried out, and the resulting performance measurements, comparing that of an OpenMP
based implementation running on multiple cores of a single CPU. 
A brief overview of this implementation will also be given, referring to [REF] for further details.

\section{Splotch Overview}
\label{sect:overview}

This section gives a brief overview to the Splotch algorithm, along with a description of the MPI and OpenMP related 
functionality for reference as these are key to the development of the MIC implementation.

Splotch is a pure C++ algorithm for generating 2 dimensional images from particle based datasets. It has no reliance 
on external libraries, minus those required for parallelism or particular file formats e.g. HDF5 [ref], and can be 
compiled from a downloadable self-contained tar-ball with a suitable makefile, use of preprocessor definitions and 
makefile switches allows compilation for a variety of hardware environments. The main stages of the Splotch workflow 
can be summarised as:

\begin{quote} %quote for indent
\begin{description}
  \item[Data Load] \hfill \\
  Source data is read using an appropriate reader and stored in the Splotch particle structure, various readers are 
  implemented for data types such as HDF5, RAMSES, and GADGET. [ref]
  \item[Processing and Rasterization] \hfill \\
   Data is preprocessed as per user requirements, performing tasks such as ranging, normalization, and applying 
   logarithms to particle attributes, amongst others. Particles are roto-translated with reference to supplied 
   camera and look-at positions along with an up-vector. Active particles, those within the view frustum, are 
   identified and assigned an RGB color value dependant on particle properties and an external colour map, 
   particles outside of the frustum are labelled inactive and not  considered further.  
  \item[Rendering] \hfill \\
  For each pixel of the image, a ray is cast along the line of sight, and contributions of all encountered particles 
  are additively accumulated. The contribution a particle may have to a particular pixel color is determined by solving 
  the radiative transfer equation:

  [Insert radiative transfer equation]

  [Description of equations components and how they affect rendering]
\end{description}
\end{quote}

\subsection{Parallel Implementations}
\label{sect:mpiopenmp}

The OpenMP additions to Splotch [ref?] that provide the foundation of the MIC algorithm are outlined in this section. 
Following this is a brief description of the MPI features of Splotch, for reference in section [MPI 
offload sect].

\subsubsection{OpenMP}
\label{sect:ompsplotch}

OpenMP is employed to parallise core sections of Splotch algorithm if working in an OpenMP enabled environment. 
The rototranslation and coloring stages consist of applying a 3D transform to the particle coordinates and assigning 
an RGB value from a color lookup table. Each thread is simply assigned a portion of the particles with which to 
perform these stages.

The rendering is slightly more complex, due to the inevitability that multiple particles will fall on the same pixels. 
Threads cannot simply draw all assigned particles anywhere in the image without the possibility of race conditions 
where multiple threads attempt to draw to a pixel at the same time. To solve this, the image is split into 100x100 pixel 
tiles. The entire particle array is preprocessed in parallel to generate a list of particle indices per tile per thread, 
which is then condensed down to a single list of particle indices per tile. Each thread is then assigned a tile, on a 
first come first serve basis, and renders all particles in the tiles designated index list. To render a particle, the 
portion of the tile that is affected is calculated, and then processed in columns of pixels. Each pixel additively 
accumulates a contribution from all particles affecting it, the contribution a particle will have on the pixel is defined 
by the equation given in section [previous section rendering]

\subsubsection{MPI}
\label{sect:mpisplotch}

MPI is incorporated into both the file readers and the core algorithm for exploiting a distributed memory system. Each 
process simply loads a subsection of the dataset, and the serial or OpenMP method is used to render to a partial image, 
which is then accumulated to a final image by the root processor. [do we actually do openmp+mpi anywhere other than 
the mic algorithm?]


\section{Splotch on the MIC}
\label{sect:micsplotch}

The MIC based implementation of Splotch uses both OpenMP and MPI in order to fully exploit the many core 
programming paradigm necessary to take advantage of this architecture. This section gives a brief overview of the 
Xeon Phi, and the steps taken to modify and optimise Splotch to make effective use of this hardware.

\subsection{Overview of the MIC Architecture}
\label{sect:mic}

The core ideal behind the MIC micro-architecture is obtaining a massive level of parallelism for high throughput 
performance in power restricted cluster environments. To this end Intel's flagship MIC product, the Xeon Phi, 
contains roughly 60 cores on a single chip, dependent on the model. Each core has access to a 512 KB private 
fully coherent L2 cache, memory controllers and the PCIe client logic provide access to up to 8 GB of GDDR5 
memory, and a bi-directional ring interconnect brings these components together. The cores are in-order, however 
up to 4 hardware threads are supported to mitigate the latencies inherent with in-order execution. The Vector 
Processor Unit (hereafter VPU) is worthy of note due to the utilisation of an innovative 512 bit wide SIMD 
capability, allowing 16 single precision (SP) or 8 double precision (DP) floating point operations per cycle, 
support for fused-multiply-add operations increased this to 32 SP or 16 DP floating point operations per cycle. 

The Xeon Phi acts as a coprocessor for a standard Intel Xeon processor connected via PCIe, and there are various 
modes of execution on a system utilising one or more Xeon Phi coprocessors. It runs the Linux operating system, 
and so can be seen as a networked node through a virtualised TCP/IP stack over the PCIe bus. This allows a user 
to log into the node, transfer a program over and run natively, or to use the coprocessor as an MPI process alongside 
the Xeon. The ability to partition subgroups of processors allows the Phi to run multiple MPI processes at the same 
time, a technique exploited in [insert subsection number]. In addition, a heterogenous approach is possible, using 
the coprocessor to accelerate a standard CPU based algorithm by offloading sections of computation to the device using 
a small series of directives available to both C++ and Fortran. 

An advantageous factor of programming for the Xeon Phi, as opposed to other accelerators commonly used in HPC 
environments such as GPUs, is the similarity of techniques used to exploit parallelism on both the Xeon Phi and 
regular Xeon processors. Algorithms that already utilise parallel paradigms involving OpenMP, MPI, Intel TBB, or 
Intel Cilk Plus can often be run on the Xeon Phi with little modification. While further tuning is necessary to 
fully take advantage of the new hardware, it is not a necessity to reimplement the entire algorithm.   


\subsection{MIC Implementation}
\label{sect:micimplementation}

The implementation targeting the MIC is based on the C++ and OpenMP version of Splotch (see section 2.1). 
While the core algorithm executes on the Xeon (referred to as the host), data is transferred to the Xeon Phi (referred 
to as the device) and the majority of the computation is offloaded using the C++ pragma extensions provided by Intel 
in a syntactic style similar to OpenMP, attention has been paid to optimising parts of the algorithm to better take 
advantage of the wide SIMD capability of the Phi.


\subsubsection{Algorithm}
\label{sect:micalgorithm}

The rendering algorithm can be broken down into a series of phases illustrated by the execution model shown in [figure x]. 
A double buffered scheme has been implemented using the ability to asynchronously transfer data via a series of signal 
and wait clauses provided by the Intel extensions. This allows to minimise overhead due to transferring data to the device 
for processing, and to facilitate the rendering of datasets potentially much larger than the memory capacity available, 
discussed further in section [section number].

\medskip
\centerline{[figure illustrating execution model]}
\medskip
A necessary overhead is incurred from the outset performing various initialisations discussed further in section 4.1. In 
addition to this, the first chunk of data must be transferred to the device before the double buffering system can begin 
to compensate for data transfer times. This initial transfer is carried out asynchronously while precomputing render 
parameters such as those used in the 3d transformation, which will be static throughout the rendering process. The render 
parameters are copied to the device, which subsequently begins rendering while the next chunk of data is transferred.

The first stage of rendering is a highly parallel 3D transform and colorize performed on a per-particle basis using 
four OpenMP threads per available core, to match the available hardware threads, equating to roughly 240 threads. This 
stage is amenable to fast parallel processing and requires only a small amount of tuning for MIC suitability.

The multistep OpenMP solution to the full rendering stage discussed in section [omp section] is modified to account for 
the larger number of threads available. Simply splitting the image into smaller tiles to allow for more threads results 
in particles affecting more tiles than previously, requiring more memory to store particle index lists per tile. To 
account for this, the following approach is adopted. 

The number of available OpenMP threads are split into groups, allocating an image buffer per group large enough to hold 
the entire resulting image. Each group is assigned a subsection of particles such that the entire array is evenly distributed. 
At this point each thread group acts independently, drawing the allocated subsection of particles to the allocated image, finally 
reducing all images into a single buffer when all particles of the current data chunk have been processed, a system 
conceptually similar to the MPI Splotch implementation discussed in section [mpi section]. The number of groups to create and 
number of threads per group can be passed in as a runtime parameter, and the chosen values should reflect consideration 
of the available number of threads and thread:core ratio. This is discussed further in section 4.1.

Subsequent to group allocation, each thread group begins independently rendering an allocated subset of particles. This 
occurs in two phases, a pre render phase and a render phase. In order to allow each thread sole access to a particular 
set of pixels, and avoid race conditions discussed previously, the image is split into 2 dimensional grid of tiles – 
the number of which is determined by a run time parameter tile\_size. The pre-render phase generates a list of particle 
indices per tile, indicating all the particles whose area of influence overlaps with the tile. In this phase each thread 
is allocated a subset of the group's allocated particles, and generates a list for each tile resulting in 
(n\_thread * n\_tile) lists. A single thread accumulates the per-thread lists to attain a single list per tile. 
Once all lists have been accumulated, phase two begins. In this phase each thread is allocated a tile, or subset of 
pixels, and renders all particles in the list associated with that tile. In this way pixels are not shared between 
threads and concurrent accesses are avoided.

Following the accumulation of each group-specific image into a single buffer, which is retained throughout the entire 
rendering process, the next chunk of data is processed until all particles have been rendered. This constant buffer is 
then copied back to the host for output. 
%\medskip
%\noindent



\subsection{Optimisation}
\label{sect:micoptimisation}

In order to appropriately exploit the MIC architecture and attain high performance gains, various optimisation methods 
were explored. Some more generic methods are applicable to other similar architectures, such as Intel 64 and IA-32 [REF], while 
others are specific to the Xeon Phi. 

\subsubsection{Memory Usage}
\label{sect:memusage}

Often scientific datasets being visualized are very large, it is inevitable that there will at times be more data than 
available device memory. This raises the issue of allocating and waiting for data transfer from main memory which, 
while reaching roughly 6 GB/s over the 16 channel PCIe 2.0 connection, is an additional overhead to be considered.
This is not such a large concern in a standard Intel 64 environment where frequent memory allocations, such as might be 
invoked by repetitive resizing of dynamic arrays, have little comparative impact.

Cost of dynamic memory allocation on the Phi is relatively high [REF1], in order to minimise unnecessary allocations 
buffers are created at the beginning of the program cycle and reused throughout. Use of the MIC\_USE\_2MB\_BUFFERS 
environment variable forces any buffers over a particular size to be allocated with 2MB rather than the default 4KB 
pages, which improves data allocation rate, transfer rate and can benefit performance by potentially reducing page 
faults and TLB (translation look-aside buffer) misses [REF2]. In isolated tests using offload clauses for compiler 
managed memory allocation on the device, a single process offloading to the device and reserving large buffers 
initially allocated at approximately 4 to 4.5s per GB (with speed increasing marginally with increased allocation size). 
Setting MIC\_USE\_2MB\_BUFFERS to 64K reduced this allocation overhead to approximately 2s per GB. [are approximations okay?]

Use of the offload syntax for asynchronous computation [REF?] allows to engage a double buffered approach, 
processing the data in smaller chunks and overlapping computation and data transfer to minimise this overhead.

\subsubsection{Vectorization}
\label{sect:vectorization}

The large 512 bit width SIMD capability of the MIC architecture is addressed through vectorization carried out both 
automatically by the compiler, and manually using Intel Initial Many-Core Instructions (IMCI) [REF3] . Firstly the 
core data structure used in Splotch rendering was re-examined, and converted from an array of structures (AoS) to 
a structure of arrays (SoA). The AoS method stores a large array of structures, each containing all of the 
information pertaining to a single particle. While this is useful for encapsulation, it is unsuitable for SIMD 
processing. [REF4] [Run test with aos vs soa and show results?] This aids the compiler in automatic vectorization, 
amongst other changes such as ensuring the correct data alignment and modifying loops to be more easily vectorized 
as described in Intels Vectorization guide [REF 5] which, while providing examples for SSE [ref?], is applicable 
to IMCI as well. 

Other areas are not so simply vectorized, and must be manually optimised through use of the Intel intrinsics, 
which map directly to IMCIs. During rendering, each thread draws all particles affecting a particular subsection 
of the final image. Particles are processed by drawing columns of affected pixels, from left to right, until all 
affected pixels have been updated, illustrated in section [insert section or figure]. Drawing consists of additively 
combining a pixels current value with the RGB contribution from the current particle, which is calculated by 
multiplying the particle colour by a contribution value. In order to expedite this process up to five single 
precision particle RGB values and five contribution values are packed into two respective 512 bit vector containers. 
A third container contains 5 affected pixels, which are written simultaneously using a fused multiply add vector 
intrinsic, masked in order not to affect the final unused float value in the 16-float capable containers. A 
complex optimisation such as this is not likely to automatically occur during compiler optimisation, however 
provides a performance increase for the rendering phase of up to 22\%.

\subsubsection{MPI offload}
\label{sect:mpioffload}

The overheads in dynamic allocation and data transfer incur a significant penalty when running a single host process 
offloading to the device. Data heavy algorithms such as Splotch require a lot of transfers, and ideally would use 
as much of the device memory at a time as possible. One such solution to this problem is to run multiple host MPI 
processes, each being awarded a subsection of the device to offload to and work with. This allows each process to 
allocate a share of the memory, transfer subsections of data, compute and render the results asynchronously, thereby 
utilising as much device memory as possible while minimising overheads. This provides a noticeable performance increase 
discussed further in section [performance analysis sect], although must be implemented through a fairly unwieldy script 
to distribute hardware threads amongst processes. [final sentence too opinionated?]


\section{Performance Analysis}
\label{sect:performance}

Performance analysis consists of running a variety of tests on the available hardware, the “Dommic” facility at the 
Swiss National Supercomputing Centre, Lugano. Each node is based upon  a dual socket eight-core Intel Xeon 2670 
processor architecture running at 2.6 GHz with 32 GB of main system memory. Two Xeon Phi 5110 MIC coprocessors are 
available per node. 

Tests are carried out using an N-Body simulation performed using Gadget [REF], consisting of roughly 400 million 
particles;  ~200 million dark matter particles, ~200 million baryonic matter particles and ~10 million star particles. 
The double buffered approach to data transfer and processing discussed in section [memory usage 3.3.1] allows to 
process large sets such as this where the device memory is not sufficient to hold the entire dataset at one time. In 
addition for smaller tests a filtering system in the file reader module allows just a subset of the data to be loaded.



\subsection{Tuning}
\label{sect:tuning}

Some necessary overheads should be considered which impact the efficacy of offloading to the device for smaller datasets, 
i.e. those taking just a few seconds to process. Initialisation of the device itself can take roughly one second, this 
occurs when the first offload clause is encountered. The impact of this can be minimised by placing an empty clause near 
to the beginning of the program, in order to initialise the device while other host activity is occurring (in this case, 
while the host is reading from file). Alternatively the environment variable OFFLOAD\_INIT can be set to on\_start to 
pre-initialise all available MIC devices before the program main begins execution [ref]. 

Initialising the large number of OpenMP threads used on the device also carries a small overhead, roughly 0.3 seconds. 
This can potentially be hidden though an empty parallel block, placed within an asynchronous offload section at the 
beginning of the program. These two overheads are relatively small and may have little effect on a long running program, 
however in a program expected to take only a few seconds to run they can outweigh any performance increase gained by 
offloading to the device. It should be noted for a small program running many times over, as is the case when creating 
a Splotch animation with a relatively small dataset, these overhead costs are paid only once during the first execution 
as for repeat run-throughs Splotch execution is not actually stopped.

Various parameters of the algorithm can be tuned to find best performance. Rendering parameters such as the number of 
thread groups, and tile size (refer to section [X]), are set to optimal defaults for the test hardware based on results 
of scripted tests iterating through sets of incremental potential values however can be modified via a parameter file 
passed in at runtime.

The performance increase provided by running multiple MPI processes all offloading to subsections of the device is 
notable. When offloading to the entire device with a single process, one of the debilitating factors was filling up 
the device memory without wasting a large amount of time transferring this data. Building upon the pre-implemented MPI 
Splotch functionality [ref to sect or ref], each process is allocated a subsection of the device via arguments to the 
mpirun command, this  allows each process to independently allocate and transfer data to the device in parallel and 
fill the device in a fraction of the time. The optimal number of processes for the test hardware was identified to be 
eight, most likely due to the host socket for each device having eight-cores. [is this speculation bad?] 

\subsection{Results}
\label{sect:results}

\noindent
- Performance tests on various sizes of data compared to CPU

\section{Discussion and Conclusions}
\label{sect:conclusions}

[Discussion of results and conclusion of work]

Intel released details of their second generation Xeon Phi product, codenamed “Knights Landing”,  at the International 
Supercomputing Conference 2013 [ref]. One important factor to note is the potential to use this as a standalone CPU 
rather than a co-processor. This is worthy of further investigation as it invites the dismissal of complex and time 
consuming data transferral methods between a host and coprocessor, amongst other problems, however also risks 
introducing new issues. One of the benefits of an offloading model is that work particularly suited to a highly 
parallelised processing model can be offloaded while unsuitable work can be handled by the more powerful Xeon host 
cores, disposal of the host would remove this advantage and beg the question of how to handle work that may be 
inherently unsuitable for a many-core architecture.
 

\subsubsection{Acknowledgments.}

Acknowledge people and things here

%The following section shows a sample reference list with entries for
%journal articles \cite{jour}, an LNCS chapter \cite{lncschap}, a book
%\cite{book}, proceedings without editors \cite{proceeding1} and
%\cite{proceeding2}, as well as a URL \cite{url}.
%Please note that proceedings published in LNCS are not cited with their
%full titles, but with their acronyms!

\begin{thebibliography}{4}

\bibitem{jour} Smith, T.F., Waterman, M.S.: Identification of Common Molecular
Subsequences. J. Mol. Biol. 147, 195--197 (1981)

\bibitem{lncschap} May, P., Ehrlich, H.C., Steinke, T.: ZIB Structure Prediction Pipeline:
Composing a Complex Biological Workflow through Web Services. In: Nagel,
W.E., Walter, W.V., Lehner, W. (eds.) Euro-Par 2006. LNCS, vol. 4128,
pp. 1148--1158. Springer, Heidelberg (2006)

\bibitem{book} Foster, I., Kesselman, C.: The Grid: Blueprint for a New Computing
Infrastructure. Morgan Kaufmann, San Francisco (1999)

\bibitem{proceeding1} Czajkowski, K., Fitzgerald, S., Foster, I., Kesselman, C.: Grid
Information Services for Distributed Resource Sharing. In: 10th IEEE
International Symposium on High Performance Distributed Computing, pp.
181--184. IEEE Press, New York (2001)

\bibitem{proceeding2} Foster, I., Kesselman, C., Nick, J., Tuecke, S.: The Physiology of the
Grid: an Open Grid Services Architecture for Distributed Systems
Integration. Technical report, Global Grid Forum (2002)

\bibitem{url} National Center for Biotechnology Information, \url{http://www.ncbi.nlm.nih.gov}

\end{thebibliography}


\end{document}
