
% Template for Elsevier CRC journal article
% version 1.0 dated 13 October 2009

% This file (c) 2009 Elsevier Ltd.  Modifications may be freely made,
% provided the edited file is saved under a different name

% This file contains modifications for Procedia Computer Science

%%%%%%%%%%%%%%%%%%%%%%%%%%%%%%%%%%%%%%%%%%%%%%%%%%%%%%%%%%%%%%%%%%%%%%%%%%

%% This template uses the elsarticle.cls document class and the extension package ecrc.sty
%% For full documentation on usage of elsarticle.cls, consult the documentation "elsdoc.pdf"
%% Further resources available at http://www.elsevier.com/latex

%%%%%%%%%%%%%%%%%%%%%%%%%%%%%%%%%%%%%%%%%%%%%%%%%%%%%%%%%%%%%%%%%%%%%%%%%%

%% The '1p' and 'times' class options of elsarticle are used for Elsevier CRC
\documentclass[1p,times]{elsarticle}

%% The `ecrc' package must be called to make the CRC functionality available
\usepackage{ecrc}

%% The ecrc package defines commands needed for running heads and logos.
%% For running heads, you can set the journal name, the volume, the starting page and the authors

%% set the volume if you know. Otherwise `00'
\volume{00}

%% set the starting page if not 1
\firstpage{1}

%% Give the name of the journal
\journalname{Procedia Computer Science}

%% Give the author list to appear in the running head
%% Example \runauth{C.V. Radhakrishnan et al.}
\runauth{M. Rivi et al.}

%% The choice of journal logo is determined by the \jid and \jnltitlelogo commands.
%% A user-supplied logo with the name <\jid>logo.pdf will be inserted if present.
%% e.g. if \jid{yspmi} the system will look for a file yspmilogo.pdf
%% Otherwise the content of \jnltitlelogo will be set between horizontal lines as a default logo

%% Give the abbreviation of the Journal.
\jid{procs}

%% Give a short journal name for the dummy logo (if needed)
\jnltitlelogo{Procedia Computer Science}

%% Hereafter the template follows `elsarticle'.
%% For more details see the existing template files elsarticle-template-harv.tex and elsarticle-template-num.tex.

%% Elsevier CRC generally uses a numbered reference style
%% For this, the conventions of elsarticle-template-num.tex should be followed (included below)
%% If using BibTeX, use the style file elsarticle-num.bst

%% End of ecrc-specific commands
%%%%%%%%%%%%%%%%%%%%%%%%%%%%%%%%%%%%%%%%%%%%%%%%%%%%%%%%%%%%%%%%%%%%%%%%%%

%% The amssymb package provides various useful mathematical symbols
\usepackage{amssymb}
%% The amsthm package provides extended theorem environments
%% \usepackage{amsthm}

%% The lineno packages adds line numbers. Start line numbering with
%% \begin{linenumbers}, end it with \end{linenumbers}. Or switch it on
%% for the whole article with \linenumbers after \end{frontmatter}.
%% \usepackage{lineno}

%% natbib.sty is loaded by default. However, natbib options can be
%% provided with \biboptions{...} command. Following options are
%% valid:

%%   round  -  round parentheses are used (default)
%%   square -  square brackets are used   [option]
%%   curly  -  curly braces are used      {option}
%%   angle  -  angle brackets are used    <option>
%%   semicolon  -  multiple citations separated by semi-colon
%%   colon  - same as semicolon, an earlier confusion
%%   comma  -  separated by comma
%%   numbers-  selects numerical citations
%%   super  -  numerical citations as superscripts
%%   sort   -  sorts multiple citations according to order in ref. list
%%   sort&compress   -  like sort, but also compresses numerical citations
%%   compress - compresses without sorting
%%
%% \biboptions{comma,round}

% \biboptions{}

% if you have landscape tables
%%%\usepackage[figuresright]{rotating}

% put your own definitions here:
%   \newcommand{\cZ}{\cal{Z}}
%   \newtheorem{def}{Definition}[section]
%   ...

% add words to TeX's hyphenation exception list
%\hyphenation{author another created financial paper re-commend-ed Post-Script}

% declarations for front matter

\newcommand{\aj}{AJ}
\newcommand{\apj}{ApJ}
\newcommand{\apjl}{ApJ}
\newcommand{\apjs}{ApJS}
\newcommand{\aap}{A\&A}
\newcommand{\aaps}{A\&AS}
\newcommand{\mnras}{MNRAS}
\newcommand{\nat}{Nature}
\newcommand{\araa}{ARAA}
\newcommand{\prd}{Phys. Rev. D}
\newcommand{\pasj}{PASJ}
\newcommand{\pasp}{PASP}
\newcommand{\physrep}{Phys. Rep.}


\begin{document}

\begin{frontmatter}

%% Title, authors and addresses

%% use the tnoteref command within \title for footnotes;
%% use the tnotetext command for the associated footnote;
%% use the fnref command within \author or \address for footnotes;
%% use the fntext command for the associated footnote;
%% use the corref command within \author for corresponding author footnotes;
%% use the cortext command for the associated footnote;
%% use the ead command for the email address,
%% and the form \ead[url] for the home page:
%%
%% \title{Title\tnoteref{label1}}
%% \tnotetext[label1]{}
%% \author{Name\corref{cor1}\fnref{label2}}
%% \ead{email address}
%% \ead[url]{home page}
%% \fntext[label2]{}
%% \cortext[cor1]{}
%% \address{Address\fnref{label3}}
%% \fntext[label3]{}

\dochead{}
%% Use \dochead if there is an article header, e.g. \dochead{Short communication}



\title{High Performance Visualization of astrophyisical data using Splotch}

%% use optional labels to link authors explicitly to addresses:
%% \author[label1,label2]{<author name>}
%% \address[label1]{<address>}
%% \address[label2]{<address>}

\author{Marzia Rivi}
\ead{m.rivi@cineca.it}
\address{CINECA, Via Magnanelli 6/3, Casalecchio di Reno, Italy}

\author{Zhefan Jin}
\ead{Zhefan.Jin@port.ac.uk}
\address{...}

\author{Klaus Dolag}
\ead{kdolag@MPA-Garching.MPG.DE}
\address{...}

\author{Claudio Gheller }
\ead{c.gheller@cineca.it}
\address{CINECA, Via Magnanelli 6/3, Casalecchio di Reno, Italy}

\author{Mel Krokos }
\ead{mel.krokos@port.ac.uk}
\address{...}

\begin{abstract}

.........

\end{abstract}

\begin{keyword}
%% keywords here, in the form: keyword \sep keyword


Visualization \sep High Performance Computing \sep Astrophysics

%% PACS codes here, in the form: \PACS code \sep code

%% MSC codes here, in the form: \MSC code \sep code
%% or \MSC[2008] code \sep code (2000 is the default)

\end{keyword}

\end{frontmatter}

%%
%% Start line numbering here if you want
%%
% \linenumbers

\section{Introduction}
\label{intro}

Data represents a critical issue for scientists and, in particular, astronomers. 
Observational instruments (telescopes, satellites...) produce enormous quantities of images and information. 
Computers and numerical simulations generate huge amount of data. All these data must be stored, 
managed and analyzed. These steps require great human effort, large scale facilities and efficient, powerful tools.  
Knowledge-discovery and data mining software often offer an efficient
instrument for automatically exploring large volumes of data. However, the most immediate,
intuitive and effective way to discover and understand the properties of data
is provided by {\it visualization}. Visualization leads the user to quickly
focus on the interesting characteristics of a system. It allows
 different properties of the data to be shown simultaneously, and it can allow the user 
to navigate inside the data. Visualization is the most natural way of finding correlations, 
similarities and patterns. It is an effective method for
selecting special regions or features, on which the researcher can concentrate his/her
efforts and the subsequent quantitative analysis. Finally, visualization is 
useful to get a quick look at an ongoing processes, e.g. a numerical simulation, 
in order, for instance, to monitor it and to understand if its behavior is as expected or if
something is going wrong and, in that case, to correct it saving time and resources.\\
For these reasons, the astronomical community has always dedicated special
attention to the growth of graphical and visualization tools, driving their evolution or even being
directly involved in the development of many of them.
At present, the most popular software for astronomers can be subdivided
into two main categories: tools for image display and processing and tools for
plotting data. Notable among the former are IRAF, by NOAO; ESO-MIDAS, by the 
European Southern Observatory; SaoImage, by the Smithsonian Astrophysical Observatory and
GAIA, by ESO. Many other tools are available, but we refer to dedicated surveys
for a complete list. Gnuplot and SuperMongo are 
popular applications adopted for 2D data plots. A more sophisticated solution 
is represented by IDL, by ITT Visual Information, which is characterized
by a large library of functions specifically developed for astrophysics. 

The last generation 
of tools, like ParaView (cite), VisIVO (cite), VisIT (cite), Aladin
\cite{Aladin} have been developed to overcome the limits
and the barriers of traditional software by exploiting the latest technological 
opportunities. Again, for a complete list, we refer to specific surveys. 

In this paper we present the High Performance Computing implementation of
Splotch, a ray-tracing software designed to visualize data coming from particle 
based computer simulations. Cosmological N-Body simulations represents a paradigmatic
example of such applications. They account for some of the largest simulations in astrophysics, 
as, for instance, the MillenniumII simulation (cite), which describe the evolution 
of a meaningful fraction of the universe by means of 10 billions of fluid elements
(particles) interacting via gravitational forces. Each output of the MillenniumII simulation
is about 200 Gbyte of data representing the 3D position and the velocity of the particles.
The manipulation of this dataset is feasibile only if high performance computing (HPC)
resources are adopted. 

The sequential version of Splotch (cite) has  been re-designed and extended in order
to exploit both multiprocessor and multi-GPUs architectures currently available. 
A Single Instruction Multiple Data (SIMD) approach has been adopted for the parallelization
of the Splotch core algorithms, adopting the MPI library to support distributed multicore systems, 
OpenMP for shared memory nodes and CUDA for exploiting NVIDIA GPU systems. The different 
solutions can be used jointly on hybrid architectures. The details and
the benchmarks of such high performance implementation will be presented in the following sections.  


\section{The Splotch Algorithm}
\label{splotch}


\section{Parallel Implementation}
\label{parallel}

The parallelization strategy of the Splotch code is based on a SIMD approach. This consists in 
distributing the data in a balanced way between the different computing elements.
Each computing element performs the same operations on its subset of data contributing 
to the final (unique) image. The details of the parallelization are given in the next sections.

\subsection{MPI-OpenMP Implementation}
\label{mpi}

\subsection{CUDA Implementation}
\label{cuda}

\section{Benchmarks}
\label{bench}




%% The Appendices part is started with the command \appendix;
%% appendix sections are then done as normal sections
%% \appendix

%% \section{}
%% \label{}

%% References
%%
%% Following citation commands can be used in the body text:
%% Usage of \cite is as follows:
%%   \cite{key}         ==>>  [#]
%%   \cite[chap. 2]{key} ==>> [#, chap. 2]
%%

%% References with BibTeX database:

\bibliographystyle{elsarticle-num}
\bibliography{<your-bib-database>}

%% Authors are advised to use a BibTeX database file for their reference list.
%% The provided style file elsarticle-num.bst formats references in the required Procedia style

%% For references without a BibTeX database:

% \begin{thebibliography}{00}

%% \bibitem must have the following form:
%%   \bibitem{key}...
%%

% \bibitem{}

% \end{thebibliography}

\end{document}

%%
%% End of file `procs-template.tex'. 
