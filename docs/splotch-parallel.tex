\input{splotch_include.tex}

\title{High Performance Visualization of astrophyisical data using Splotch}

%% use optional labels to link authors explicitly to addresses:
%% \author[label1,label2]{<author name>}
%% \address[label1]{<address>}
%% \address[label2]{<address>}

\author{Marzia Rivi}
\ead{m.rivi@cineca.it}
\address{CINECA, Via Magnanelli 6/3, Casalecchio di Reno, Italy}

\author{Zhefan Jin}
\ead{Zhefan.Jin@port.ac.uk}
\address{School of Creative Technologies, University of Portsmouth, Portsmouth, United Kindom}

\author{Klaus Dolag}
\ead{kdolag@MPA-Garching.MPG.DE}
\address{Max-Planck-Institut f\"ur Astrophysik, Karl-Schwarzschild Strasse 1, Garching bei M\"unchen, Germany}

\author{Claudio Gheller }
\ead{c.gheller@cineca.it}
\address{CINECA, Via Magnanelli 6/3, Casalecchio di Reno, Italy}

\author{Mel Krokos }
\ead{mel.krokos@port.ac.uk}
\address{...}

\begin{abstract}

......... 

\end{abstract}

\begin{keyword}
%% keywords here, in the form: keyword \sep keyword


Visualization \sep High Performance Computing \sep Astrophysics

%% PACS codes here, in the form: \PACS code \sep code

%% MSC codes here, in the form: \MSC code \sep code
%% or \MSC[2008] code \sep code (2000 is the default)

\end{keyword}

\end{frontmatter}

%%
%% Start line numbering here if you want
%%
% \linenumbers

\section{Introduction}
\label{intro}

Data represents a critical issue for scientists and, in particular, astronomers. 
Observational instruments (telescopes, satellites...) produce enormous quantities of images and information. 
Computers and numerical simulations generate huge amount of data. All these data must be stored, 
managed and analyzed. These steps require great human effort, large scale facilities and efficient, powerful tools.  
Knowledge-discovery and data mining software often offer an efficient
instrument for automatically exploring large volumes of data. However, the most immediate,
intuitive and effective way to discover and understand the properties of data
is provided by {\it visualization}. Visualization leads the user to quickly
focus on the interesting characteristics of a system. It allows
 different properties of the data to be shown simultaneously, and it can allow the user 
to navigate inside the data. Visualization is the most natural way of finding correlations, 
similarities and patterns. It is an effective method for
selecting special regions or features, on which the researcher can concentrate his/her
efforts and the subsequent quantitative analysis. Finally, visualization is 
useful to get a quick look at an ongoing processes, e.g. a numerical simulation, 
in order, for instance, to monitor it and to understand if its behavior is as expected or if
something is going wrong and, in that case, to correct it saving time and resources.\\
For these reasons, the astronomical community has always dedicated special
attention to the growth of graphical and visualization tools, driving their evolution or even being
directly involved in the development of many of them.
At present, the most popular software for astronomers can be subdivided
into two main categories: tools for image display and processing and tools for
plotting data. Notable among the former are IRAF \cite{iraf}, by NOAO; ESO-MIDAS \cite{midas}, by the 
European Southern Observatory; SaoImage \cite{sao}, by the Smithsonian Astrophysical Observatory and
GAIA, by ESO. Many other tools are available, but we refer to dedicated surveys
for a complete list. Gnuplot and SuperMongo are 
popular applications adopted for 2D data plots. A more sophisticated solution 
is represented by IDL, by ITT Visual Information \cite{idl}, which is characterized
by a large library of functions specifically developed for astrophysics. 

The last generation 
of tools, like ParaView \cite{paraview}, VisIVO \cite{visivo}, VisIT \cite{visit}, Aladin
\cite{aladin}, have been developed to overcome the limits
and the barriers of traditional software by exploiting the latest technological 
opportunities. Again, for a complete list, we refer to specific surveys. 

In this paper we present the High Performance Computing implementation of
Splotch \citep{2008NJPh...10l5006D}, a ray-tracing software designed to visualize data coming from particle 
based computer simulations. Cosmological N-Body simulations represents a paradigmatic
example of such applications. They account for some of the largest simulations in astrophysics, 
as, for instance, the MillenniumII simulation \citep{2009MNRAS.398.1150B}, which describe the evolution 
of a meaningful fraction of the universe by means of 10 billions of fluid elements
(particles) interacting via gravitational forces. Each output of the MillenniumII 
simulation is about 400 Gbyte of data representing the 
3D position and the velocity of the particles, their ID and some additional, calculated
properties like the local smoothing length (e.g. distance wich contains 32 neighbours),
local density or local velocity dispersion. The manipulation of this dataset is feasibile 
only if high performance computing (HPC) resources are adopted. 

The sequential version of Splotch \cite{2008NJPh...10l5006D} has been re-designed and extended in order
to exploit both multiprocessor and multi-GPUs architectures currently available. 
A Single Instruction Multiple Data (SIMD) approach has been adopted for the parallelization
of the Splotch core algorithms, adopting the MPI library to support distributed multicore systems, 
OpenMP for shared memory nodes and CUDA for exploiting NVIDIA GPU systems. The different 
solutions can be used jointly on hybrid architectures. The details and
the benchmarks of such high performance implementation will be presented in the following sections.  


\section{The Splotch Algorithm}
\label{splotch}

\begin{figure}
\begin{center}
\includegraphics[width=0.49\textwidth]{millenium2_veldisp.pdf}
\includegraphics[width=0.49\textwidth]{millenium2_vel.pdf}
\end{center}
\caption{Shown are vizualisation of the MilleniumII simulation \citep{2009MNRAS.398.1150B} using
the velocity dispersion (left panel) and the three dimensional velocity (right panel) of the
particles for the color transfere function. The inlay shows a zoom on one the most rich structure
inside the simulation.}\label{mil2}
\end{figure}

The rendering algorithm of {\tt Splotch} is generally designed to handle
point-like particle distributions. Such tracer particles can be smoothed
to obtain a contineous field, most commonly the $B_2$-Spline \citep{1985A&A...149..135M}
\begin{equation}
   W(x,h)=\frac{8}{\pi h^3}\left\{\begin{array}{ll}
      1 - 6 \left(\frac{x}{h}\right)^2 + 6 \left(\frac{x}{h}\right)^3 \;\;& 0 \le \frac{x}{h} < 0.5 \\
      2 \left(1 - \frac{x}{h}\right)^3                              & 0.5 \le \frac{x}{h} < 1 \\
      0                                                             & 1 \le \frac{x}{h} \\
   \end{array} \right. , \label{SPH:kern}
\end{equation}
where $h$ is the local smoothing length, which is typically defined in a way
that every particle overlaps with $\approx 32$ neighbors. Therefore, the rendering is
based on the following assumptions:

\begin{itemize}
\item The contribution to the matter density by every particle can
be described as a Gaussian distribution of the form 
$\rho_p(\vec r)=\rho_{0,p}\exp(-r^2/\sigma_p^2)$.\footnote{Note that the
$b_2$-Spline kernel used in SPH has a shape very similar to the Gaussian distribution.} In
practice, it is much more handy to have a compact support of the
distribution, and therefore the distribution is set to zero at a given
distance of $f\cdot\sigma_p$. Following the approach often used to vizualice 
cosmological simulation we choose $f$ in such a way that
$f\cdot\sigma_p$ is related to the smoothing length $h$, i.e.
to fulfill $h \approx f\cdot\sigma_p$. Therefore rays passing
the particle at a distance larger than $f\cdot\sigma_p$ will be practically
unaffected by the particle's density distribution.
\item We use three ``frequencies'' to describe the red, green and blue
components of the radiation, respectively. These are treated independently.
\item The radiation intensity $\bf{I}$\footnote{Here we treat all
intensities as vectors with r,g and b components.} along a ray through the simulation
volume is modeled by the well known radiative transfer equation
\begin{equation}
\frac{d\bf{I}(x)}{dx}=(\bf{E}_p-\bf{A}_p\bf{I}(x))\rho_p(x),
\end{equation}
which can be found in standard textbooks like \citet{1991par..book.....S}.
Here, $\bf{E}_p$ and $\bf{A}_p$ describe the strength of radiation emission and absorption
for a given particle for the three rgb-colour components. In general it is recommended to
set $\bf{E}_p=\bf{A}_p$, which typically produces visually appealing images; for special
effects, however, independent emission and absorption coefficients can be used. Specially typically
a small gray component (e.g. a small, constant addition to the rgb-components) can be added 
to the absortpion of each particle ($\bf{A}_p$) to improve the three dimensional impression.
In general the coefficients can vary between particles, and are typically chosen as a function 
of a characteristic particle property. If a scalar quantity is choosen (e.g. the particle temperature, 
density, velocity dispersion, etc.), the mapping to the three components of $\bf{E}$ and $\bf{A}$ (for red, green and blue)
is typically achieved via a transfere function, realized by a colour look-up table or palette, which can
be provided to the ray-tracer as an external file to allow a maximum of flexibility. If a
vector quantity is chosen (r.g. velocity, magnetic field, etc.), the three components of the vectors
can be maped to the three components of $\bf{E}$ and $\bf{A}$ (for red, green and blue). Additional 
to the color, the intensity of each particle can be aditional modulated proportional to another
scalar property (e.g. density, etc.).
\end{itemize}

Assuming that absorption and emission are homogeneously mixed it
can be shown that, whenever a ray traverses a particle, its intensity
change is given by
\begin{equation}
\label{i_change}
\bf{I}_{\mbox{after}}=(\bf{I}_{\mbox{before}}-\bf{E}_p/\bf{A}_p)
\exp(-\bf{A}_p\int_{-\infty}^\infty\rho_p(x)dx) + \bf{E}_p/\bf{A}_p
\end{equation}
The integral in this equation is given by
$\rho_{0,p}\sigma_p\exp{(-d_0^2/\sigma_p^2)}\sqrt{\pi}$, where $d_0$
is the minimum distance between the ray and the particle center.

Under the assumption that the particles do not overlap, the intensity
reaching the observer could simply be calculated by applying this formula to
all particles intersecting with the ray, in the order of decreasing distance
to the observer. In reality, of course, the particles do overlap, but since
the relative intensity changes due to a single particle can be assumed to be
very small, this approach can nevertheless be used in good approximation.

Figure \ref{mil2} shows a vizualisation example of a large simulation 
containing 10 billion particles.

\section{Parallel Implementation}
\label{parallel}

The Splotch code workflow consists in a number of steps which can be summarized as follows:

\noindent 1. Read data from one or more files;

\noindent 2. process data for rescaling, normalization etc.;

\noindent 3. render data;

\noindent 4. save the final image.

All these steps can be parallelized according to the same strategy, based on a SIMD approach. 
This consists in 
distributing the data in a balanced way between the different computing elements.
Each computing element performs the same operations on its subset of data contributing 
to the final (unique) image. 

The parallel implementation has been accomplished using different approaches, suitable 
for different hardware architectures and software environments. The MPI library \cite{mpi} 
has been used to define the overall data and work distribution characterization. Data are
read in chuncks of the same (or similar) size from each processor in the MPI pool. Then 
the same work is performed for steps 2 and 3 on the chuncks by each processor. Finally,
the final image is generated and saved only in the master processor, since this is a 
light task, which does not require any kind of parallel implementation. The work accomplished
in steps 2 and 3 can be further split, exploiting multicore shared memory processors or Graphic Cards.
In the first case, an OpenMP \cite{openmp} based approach provides the necessary parallelization tools. 
In the second case, the CUDA \cite{cuda} programming language has been adopted. The different approaches
can be used separately, if the available computing system fits only one of the available 
configurations, or jointly. For instance, on a single core PC with an NVIDIA graphic card, 
only the CUDA based parallelization strategy can be activated and exploited. On a multicore
RVN \cite{rvn} node both MPI and CUDA can be used. This makes the parallel Splotch code extremely 
flexible, portable, efficient and scalable.


\subsection{MPI Implementation}
\label{mpi}

Once the data is properly distributed among the processors, all the remaining operations 
are performed locally and further communication is not needed, until the generation
of the final result. Each MPI process uses the assigned 
data to produce its own partial image. At the end, all the partial contributes are merged by means of a 
collective reduction operation producing the final image. 

The data load stage represents the crucial step for balancing the workload and for fast reading 
data from the disk. In fact, from the results presented in section x.x, it is clear how,
as the data size grows,
the data load process tends to be the most demanding section of the code. For this reason,
we have paied specific attention to the effective implementations of this functionality.

The adoption of MPI I/O based functions, represents the ideal solution for obtaining 
a high-performance and scalable read utility. 
With this approach each process has a different fileview of a single file that allows simultaneous and collective
writing/reading of noncontiguous interleaved data. A view defines what data are visible to each process and 
consists of: an offset, measured in bytes from the beginning of the file; an elementary type (etype) that 
defines the unit of data access and positioning within the file; and a template for accessing the file (filetype)
consisting in a number of etypes and holes (which must be a multiple of the etype size). 
The basic filetype is repeated again and again, tiling the file, and creating regions of allowed access 
(where etypes are defined), and no access (where holes are defined).

Our implementation assumes that data are organized in the input file according to a block structure, 
where each block contains a single information for all $N$ particles. Therefore we have as many contiguous blocks
as the number $n$ of properties given for each particle, and we can see them as a 2-dimensional array $A$ 
of $n \times N$ float elements. 
Then we have defined the MPI I/O filetype as a simple 2-dimensional subarray of $A$ 
of size $n \times N/nprocs$.  
 
In order to support high performance computing environments where MPI I/O is not available, 
we have also provided two standard MPI binary readers that can respectively 
read binary data files written both in table and block formats. 
In both cases data to read are equally distributed among processes and simultaneously each one reads  
their own portion of data by a direct access operation. 
The block reader executes one single read of size $N/nprocs$ for each block of data, 
while the tabular one reads $N/nprocs$ rows containing all $n$ quantities related to a single particle.

In all readers an endianism swap is also managed when required.   



\subsection{CUDA Implementation}
\label{cuda}
Nowadays GPUs have significantly benefited scientific applications by its fast ALU 
capability and stream data-parallel organization. With the introduction of
 Computer Unified Device Architecture (CUDA), graphics high-performance 
 ability can be accessed by a C language interface, allowing developers to bypass 
 the difficulty of fitting a general algorithm onto fixed graphics primitives. 
To obtain the computing horsepower of GPUs, CUDA technology has been applied
to Splotch.

CUDA programs runs on GPUs with a Single Instruction Multiple Thread (SIMT)
mode. Every instruction issue time, a multiprocessor will execute one common instruction 
for multiple threads. But if threads diverge via a data-dependent 
conditional branch, they will be executed serially which will cause a 
slow-down in the execution. In the worst case for a warp \cite{cuda} there could 
be only one thread running at each time which causes the multiprocessor
 actually degenerating to a single-processor. For this reason the parallel strategy of 
 the MPI implementation in section 3.1 is not suitable here as CUDA threads tends to 
 diverge wildly after running through multiple particles. And the graphics memory 
required for image buffer in MPI implementation will limit the scale 
of parallel threads which makes a full horsepower from CUDA unavailable.

With those considerations we set the granularity of parallel task to one particle, which 
means that each CUDA thread will take care of one particle's drawing, 
going through all the steps including ranging, transformation, colorization and 
rendering. As the number of particles for cosmological dataset is often large, 
say millions of, the multiprocessors will be fully fed. And in most of the steps the 
granularity of the tasks are stable among particles which is good for
parallel computing until the rendering step, where some specially treatment will be applied.


\begin{figure}
\begin{center}
\includegraphics[width=0.50\textwidth]{cu_splotch_process.png}
\end{center}
\caption{Process of CUDA Splotch.}
\end{figure}

The overall process is illustrated in Figure 2. The red arrows show the 
execution of the processing steps. On the device side there are 
many arrows going together as CUDA threads running in parallel. In the
 Range step particle data are normalized and some global information is calculated, 
 like the maximum and minimum value of certain attributes. We let CPU handle
  the calculation of this as well as particle sorting after transformation.

After geometry transformation the area, or the location of the sub-image that each 
particle will influence is determined, and the empty sub-images will later be filled 
in the render step. Depending on the smoothing length of the particles and the 
camera settings the sizes of the sub-images range from 1*1 to the size of the 
final image, say 1000*1000, so the granularity of computing tasks in render step 
could be quite unbalanced. We found this unbalancing compromises speed, as shown in Table 3.

To solve that problem we imposed a new step on host, showing as 'Split Particles' 
in figure 2, to divide big tasks into smaller, average ones. Any particle
carrying a sub-image whose pixel number is larger than a barrier value will be 
splitted into multiple ones, each carrying a part of the pixels. The barrier value 
can be configured outside. In practical we found a number around the width 
or height of the final image, say 1024 usually works well.

There are of course cost for doing the splitting. The number of particles is 
increased which demands more thread issuing. Running the splitting algorithm itself
as well as the memory copying between host and device will also cost some time. 
Besides all these costs test results show that splitting brings speed increasing, as show in Table 3.

The output of rendering for the particles is stored in a one-dimension fragment buffer. 
After filled at the device-render step, fragment buffer is copied to host
 and combined to the final image by CPU. Here we found that the asynchronism of 
 device calling can be used to let device and host work in parallel. Say, when
rendering particle chunk N, combine the result of chunk N-1, the algorithm is listed below.

\begin{verbatim}
while  ( not all particles rendered )
{
       find the i-th subset S(i) in particle array that just uses up 
       fragment buffer;
A:     call device to render S(i);
       if  ( S(i) is not the first subset starting from index 0 )
       { 
B:          combine F(i-1), the output of S(i-1) in fragment buffer;
       }
C:     copy fragment buffer from device to host;
       if  ( S(i) is the last subset ending with index N, last index of
             particle array)
       {
             combine F(i);
       }
       i++;
}
\end{verbatim}


Instruction A is actually executed on device, which runs in parallel with 
instruction B, and the two paths merge at instruction C where they all have finished
. From test result we found that the combine time is mostly hidden by the render time.

In the future there are ways that worth trying to optimize CUDA Splotch. There are suggested, standard approaches
that can be tried to optimize CUDA Splotch, like unrolling short loops to optimize the control flow, using 
local/shared memory to accelerate data fetching and so on. As a parallel approach load-balancing is always an issue
which needs taking care of. For a system running CUDA, the load balance between CPU and GPU, and that between GPUs
should all be considered in order to obtain the full power of the hardware.



\section{Benchmarks}

The parallel Splotch code has been tested on different architectures and datasets, in order to 
exploit the various parallel approaches and to emphasize different characteristics of the software.

Taking advantage of the high portability of the code, we could perform our tests on a 5000 cores UNIX-AIX 
SP6 system, a cluster of 168 Power6 575 compute nodes with 32 cores per node and a memory of 128GB/node
(we will refer to this platform as SP6) and a Windows XP Machine with a CPU of Intel Xeon X5482 3.2 GHz and
2 GPUs of NVIDIA Quadro FX 5600 (indicated as Win). 
In this way, we can test the usage of Splotch on computing systems 
of different sizes and different target applications, from a standard PC (Win), where small-medium
data files can be used, up to 
a High Performance Computing platform where the huge and complex datasets can be processed.

Five different dataset have been used for the benchmarks. The first four are derived from 
the final output of a cosmological N-Body simulation, run using 870 millions of particles 
(fluid elements), characterized by their 3D coordinates and velocities, their mass density 
and the smoothing length, which defines the size of the region influenced by the properties
of each particle (see equation \ref{SPH:kern}). Subsets of 1, 10, 100 millions particles have been
randomly extracted from the whole dataset, producing three files of increasing size, indicated as
block1M, block10M and block100M. A fourth file, block870M contains the entire data product. 
Such variety of data sizes is necessary to match the memory requirements of the different available 
computing systems used used for the tests.

A further test has been performed using the final output of the MillenniumII simulation. 
This is our final reference benchmark, which consists in the visualization of a 10 billion
particle dataset.

All the data files are pure binaries, organized so that different quantitites are stored one after
the other in the file (e.g. $x$ coordinates of all particles are stored first, followed by $y$ 
coordinates and so on). In this way, we can easily adopt our MPI-IO based
reader. Only in the case of the MillenniumII data, we test also the native 
Gadget2 \cite{gadget} file format, using a integrated parallel version of the Gadget2 reader.

A summmary of the benchmarks is presented in table XXX. 

\begin{table}
\caption{Features of the datasets used in the benchmarks}
\begin{tabular}{|l|l|l|}
\hline
Label & 	N. particles & 	File size (Bytes)  \\
\hline
1M   & 	1000000   & 24000000 \\
\hline
10M  & 	10000000  & 240000000 \\
\hline
100M & 	100000000 & 2400000000 \\
\hline
850M & 	882739509 & 21185748216 \\
\hline
M-II & 	10000000000 & 	443456159744 \\
\hline
\end{tabular}
\end{table}


\subsection{MPI Benchmarks}

Table XX shows the test results on SP6 for dataset block 100M and block 870M.
Power6 cores can schedule two processes or threads in the same clock cycle, 
and it is possible to use a single core as two virtual CPUs. This mode of Power6 
is called Simultaneous Multi Threading (SMT).  This mode notably improves performance 
of process and rendering phases, reducing the execution time by 30%, but in I/O operations 
this reduction is lower and decreases as the chuncks of data to read grows.

Comparing MPI I/O reader with standard block reader, we see that improvements due to MPI2 functions 
appears when processes read large chunks of data. So good performances are expected as size of data increases.
Moreover when number of processes is high MPI I/O reader seems to perform and scale better than the other one.
   

\subsection{CUDA Benchmarks}

\begin{table}
\caption{Result of CUDA Splotch for benchmark 1M and 10M.}
\begin{tabular}{|l|l|l|l|l|}
\hline
	& setup/read(s) & 	drawing(s) & 	write(s) & 	total(s) \\
\hline
block 1M without CUDA & 	1.1259 & 	0.6982 & 	0.1860 & 	2.0103 \\
\hline
block 1M with CUDA & 	1.0916 & 	0.8411 & 	0.1840 & 	2.1170 \\
\hline
block 10M without CUDA & 	10.7064 & 	6.1376 & 	0.1842 & 	17.0284 \\
\hline
blocm 10M with CUDA & 	10.1411 & 	4.8261 & 	0.1865 & 	15.1537 \\
\hline
\end{tabular}
\end{table}

Table 2 shows the test result on Win for dataset block 1M and block 10M. For the
smaller dataset block 1M we can see that the speed with CUDA is slower than that without
it, because for small tasks the cost for applying CUDA can not be ignored, 
such as initialize CUDA runtime, data copying between host and device and so on. For block
10M we can see using CUDA gained us 21.3\% of speedup, here we define speedup as

speedup = (TimeWithoutCUDA - TimeWithCUDA)/TimeWithoutCUDA

CUDA(GPU) Splotch usually works better when handling the datasets
that generate visually appealing images like Figure 3, where drawing especially rendering
is a major work load. Table 3 shows some test result of such datasets. 
From the result we can see that CUDA provides a speedup to Splotch, and 
particle dividing helped improving the performance.


\begin{table}
\caption{Result of CUDA Splotch for small, middle and big datasets. The number of particle for the 
datasets, small=370,852, middle=2,646,991, big=16,202,527}
\begin{center}
\begin{tabular}{ | l || l | l || l | l | p{2cm} |}
\hline  
   &	CPU	& CPU	& GPU	& GPU	& GPU \\
\hline
  data & drawing time(s) & total time(s) & drawing time(s) & total time(s) & drawing time without splitting(s) \\
\hline  
  small &	12.6419	& 13.0144	& 7.3885	& 7.7668	& 12.4894 \\
\hline
  middle &	18.5443 &	19.7190 &	11.3855 &	12.7118 &	16.6905 \\
\hline
  big	& 31.8976 &	39.3234 &	19.7792 &	27.0321	& 24.6305\\
\hline
\end{tabular}
\end{center}
\end{table}

\begin{figure}
\begin{center}
\includegraphics[width=1.0\textwidth]{cu_images.png}
\end{center}
\caption{Rendering result for small, middle and big datasets.}
\end{figure}


\begin{figure}
\begin{center}
\includegraphics[width=0.5\textwidth]{t_cpu.pdf}
\end{center}
\caption{Shown is the scaling of the CPU time (total wallclock time substracting the time needed for 
readind and writing) with the number of MPI threads used for vizualisng 10 billion particles with an final
image size of 3200x3200 pixels. The dashed line indicates the expectation for an ideal scaling. The test was 
performed on a {\it Power6} architecture. The diamonds are runs using one task per core, the triangles indicate
using two tasks per core (a special feature of the {\it Power 6} architecture), where we measure a speedup of 
a factor of 1.4.}\label{cpu_scaling}
\end{figure}

\begin{figure}
\begin{center}
\includegraphics[width=0.5\textwidth]{t_read.pdf}
\end{center}
\caption{Shown is the scaling of the allclock time needed to read the needed data 
(position, velocity and smoothing length) for the 10 billion particles of the simulation
as function of parallel reading tasks. The dashed line indicates the expectation for an 
ideal scaling. The horizontal lines indicate time expected for and IO throughput of 300 Mbype/sec,
1 Gbyte/sec and 3 Gbyte/sec. The read datapoint indicate the reading speed obtained if naively 
individual data elements are streamed instead of reading large blocks in juncks.}\label{read_scaling}
\end{figure}

\subsection{Visualization of the MillenniumII run}
\label{mII}



\section{Conclusions}
\label{conclusions}

Visualization is a powrful tool to explore in an immediate and 
effective way data, expecially large data, focusing on interesting or special features
which would be much harder to identify with a blind, systematic approach. 
Obviously, visualization cannot supersede quantitative analysis, however it helps in
understending the meaning and the content of complex digital information.

The continuously increasing data size and complexity requires more and more 
sophisticated visualization tools, able to appropriately render the information 
hidden in the files, which can process enormous data volumes, such those produced 
by ultimate scientific experiments or computer simulations.

Splotch matches both requirements. It produces high quality image output,
based on a ray tracing algorithm specifically designed to exploit at best the 
properties of point-like data distributions. Furthermore, it can run on a large 
variety of high performance computing architectures, thanks to its hybrid
parallel implementation, which can take advantage of multi-processors system, adopting
an MPI based approach, multi-core shared memory processors, exploiting OpenMP,
GPUs, relying on CUDA. This allows to achieve high performance overcoming the typical
memory barriers posed by small personal systems, commonly adopted for visualization.
Finally, Splotch is written in ANSI C++ and is compeltely self-contained (it does not
require any complementary library, apart from MPI, OpenMP and CUDA stuff). 
This make the code easily portable and compilable over a large number of different 
architectures and operating systems. The next steps will be that of porting and 
running Splotch on hybrid system, multiprocessor computers with GPUs, exploiting 
both MPI and CUDA on the same dataset.

We have tested Splotch on datasets of different sizes. In particular we manage to 
visualize the largest cosmological simulation currently available, the MillenniumII run, 
processing 10 billions particle in a single shot.

\section*{Acknowledgments}
We would like to thank Mike for providing the data of the MilleniumII simulations 
on which we conducted our perfgomance tests. K.~D.~acknowledges the
supported by the DFG Priority Programme 117. Finally, we acknowledge 
HPC Europa 2 project, funded by the European Commission - 
DG Research in the Seventh Framework Programme under grant agreement n� 228398, 
for the committed resources.

\section*{References}


%% References with BibTeX database:
\begin{thebibliography}{00}

\bibliographystyle{elsarticle-num}
\bibliography{master.bib}

%% Authors are advised to use a BibTeX database file for their reference list.
%% The provided style file elsarticle-num.bst formats references in the required Procedia style

%% For references without a BibTeX database:



\bibitem{iraf} http://iraf.noao.edu/

\bibitem{midas} http://www.eso.org/sci/data-processing/software/esomidas/

\bibitem{sao} http://tdc-www.harvard.edu/software/saoimage.html

\bibitem{idl} http://www.ittvis.com/

\bibitem{paraview} http://www.paraview.org/

\bibitem{visivo}Visualization, Exploration and Data Analysis of Complex Astrophysical Data,
M. Comparato, U. Becciani, A. Costa, B. Garilli, C. Gheller, B. Larsson, J. Taylor, 2007, 
The Publications of the Astronomical Society of the Pacific, Volume 119, Issue 858, pp. 898-913.

\bibitem{visit} https://wci.llnl.gov/codes/visit/

\bibitem{aladin} http://aladin.u-strasbg.fr/

\bibitem{2008NJPh...10l5006D}Splotch: Visualizing Cosmological Simulations. K.Dolag, 
M. Reinecke, C.Gheller, S. Imboden, 2008, New Journal of Physics, Volume 10, Issue 12, pp. 125006

\bibitem{mpi} http://www.mpi-forum.org/

\bibitem{openmp} http://openmp.org/

\bibitem{cuda} http://www.nvidia.com/object/cuda\_home.html

\bibitem{rvn} http://ibm-deep-computing-visualization-rvn-end.software.informer.com/

\bibitem{gadget} Springel V 2005 Monthly Notices of the Royal Astronomical Society 364, 1105-1134


\end{thebibliography}

\end{document}

%%
%% End of file `procs-template.tex'. 